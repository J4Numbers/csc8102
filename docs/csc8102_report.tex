
%% bare_jrnl_compsoc.tex
%% V1.4b
%% 2015/08/26
%% by Michael Shell
%% See:
%% http://www.michaelshell.org/
%% for current contact information.
%%
%% This is a skeleton file demonstrating the use of IEEEtran.cls
%% (requires IEEEtran.cls version 1.8b or later) with an IEEE
%% Computer Society journal paper.
%%
%% Support sites:
%% http://www.michaelshell.org/tex/ieeetran/
%% http://www.ctan.org/pkg/ieeetran
%% and
%% http://www.ieee.org/

%%*************************************************************************
%% Legal Notice:
%% This code is offered as-is without any warranty either expressed or
%% implied; without even the implied warranty of MERCHANTABILITY or
%% FITNESS FOR A PARTICULAR PURPOSE! 
%% User assumes all risk.
%% In no event shall the IEEE or any contributor to this code be liable for
%% any damages or losses, including, but not limited to, incidental,
%% consequential, or any other damages, resulting from the use or misuse
%% of any information contained here.
%%
%% All comments are the opinions of their respective authors and are not
%% necessarily endorsed by the IEEE.
%%
%% This work is distributed under the LaTeX Project Public License (LPPL)
%% ( http://www.latex-project.org/ ) version 1.3, and may be freely used,
%% distributed and modified. A copy of the LPPL, version 1.3, is included
%% in the base LaTeX documentation of all distributions of LaTeX released
%% 2003/12/01 or later.
%% Retain all contribution notices and credits.
%% ** Modified files should be clearly indicated as such, including  **
%% ** renaming them and changing author support contact information. **
%%*************************************************************************


% *** Authors should verify (and, if needed, correct) their LaTeX system  ***
% *** with the testflow diagnostic prior to trusting their LaTeX platform ***
% *** with production work. The IEEE's font choices and paper sizes can   ***
% *** trigger bugs that do not appear when using other class files.       ***                          ***
% The testflow support page is at:
% http://www.michaelshell.org/tex/testflow/


\documentclass[10pt,journal,compsoc]{IEEEtran}
%
% If IEEEtran.cls has not been installed into the LaTeX system files,
% manually specify the path to it like:
% \documentclass[10pt,journal,compsoc]{../sty/IEEEtran}





% Some very useful LaTeX packages include:
% (uncomment the ones you want to load)


% *** MISC UTILITY PACKAGES ***
%
%\usepackage{ifpdf}
% Heiko Oberdiek's ifpdf.sty is very useful if you need conditional
% compilation based on whether the output is pdf or dvi.
% usage:
% \ifpdf
%   % pdf code
% \else
%   % dvi code
% \fi
% The latest version of ifpdf.sty can be obtained from:
% http://www.ctan.org/pkg/ifpdf
% Also, note that IEEEtran.cls V1.7 and later provides a builtin
% \ifCLASSINFOpdf conditional that works the same way.
% When switching from latex to pdflatex and vice-versa, the compiler may
% have to be run twice to clear warning/error messages.






% *** CITATION PACKAGES ***
%
\ifCLASSOPTIONcompsoc
  % IEEE Computer Society needs nocompress option
  % requires cite.sty v4.0 or later (November 2003)
  \usepackage[nocompress]{cite}
\else
  % normal IEEE
  \usepackage{cite}
\fi
% cite.sty was written by Donald Arseneau
% V1.6 and later of IEEEtran pre-defines the format of the cite.sty package
% \cite{} output to follow that of the IEEE. Loading the cite package will
% result in citation numbers being automatically sorted and properly
% "compressed/ranged". e.g., [1], [9], [2], [7], [5], [6] without using
% cite.sty will become [1], [2], [5]--[7], [9] using cite.sty. cite.sty's
% \cite will automatically add leading space, if needed. Use cite.sty's
% noadjust option (cite.sty V3.8 and later) if you want to turn this off
% such as if a citation ever needs to be enclosed in parenthesis.
% cite.sty is already installed on most LaTeX systems. Be sure and use
% version 5.0 (2009-03-20) and later if using hyperref.sty.
% The latest version can be obtained at:
% http://www.ctan.org/pkg/cite
% The documentation is contained in the cite.sty file itself.
%
% Note that some packages require special options to format as the Computer
% Society requires. In particular, Computer Society  papers do not use
% compressed citation ranges as is done in typical IEEE papers
% (e.g., [1]-[4]). Instead, they list every citation separately in order
% (e.g., [1], [2], [3], [4]). To get the latter we need to load the cite
% package with the nocompress option which is supported by cite.sty v4.0
% and later. Note also the use of a CLASSOPTION conditional provided by
% IEEEtran.cls V1.7 and later.





% *** GRAPHICS RELATED PACKAGES ***
%
\ifCLASSINFOpdf
  \usepackage[pdftex]{graphicx}
  % declare the path(s) where your graphic files are
  % \graphicspath{{../pdf/}{../jpeg/}}
  % and their extensions so you won't have to specify these with
  % every instance of \includegraphics
  % \DeclareGraphicsExtensions{.pdf,.jpeg,.png}
\else
  % or other class option (dvipsone, dvipdf, if not using dvips). graphicx
  % will default to the driver specified in the system graphics.cfg if no
  % driver is specified.
  % \usepackage[dvips]{graphicx}
  % declare the path(s) where your graphic files are
  % \graphicspath{{../eps/}}
  % and their extensions so you won't have to specify these with
  % every instance of \includegraphics
  % \DeclareGraphicsExtensions{.eps}
\fi
% graphicx was written by David Carlisle and Sebastian Rahtz. It is
% required if you want graphics, photos, etc. graphicx.sty is already
% installed on most LaTeX systems. The latest version and documentation
% can be obtained at: 
% http://www.ctan.org/pkg/graphicx
% Another good source of documentation is "Using Imported Graphics in
% LaTeX2e" by Keith Reckdahl which can be found at:
% http://www.ctan.org/pkg/epslatex
%
% latex, and pdflatex in dvi mode, support graphics in encapsulated
% postscript (.eps) format. pdflatex in pdf mode supports graphics
% in .pdf, .jpeg, .png and .mps (metapost) formats. Users should ensure
% that all non-photo figures use a vector format (.eps, .pdf, .mps) and
% not a bitmapped formats (.jpeg, .png). The IEEE frowns on bitmapped formats
% which can result in "jaggedy"/blurry rendering of lines and letters as
% well as large increases in file sizes.
%
% You can find documentation about the pdfTeX application at:
% http://www.tug.org/applications/pdftex






% *** MATH PACKAGES ***
%
%\usepackage{amsmath}
% A popular package from the American Mathematical Society that provides
% many useful and powerful commands for dealing with mathematics.
%
% Note that the amsmath package sets \interdisplaylinepenalty to 10000
% thus preventing page breaks from occurring within multiline equations. Use:
%\interdisplaylinepenalty=2500
% after loading amsmath to restore such page breaks as IEEEtran.cls normally
% does. amsmath.sty is already installed on most LaTeX systems. The latest
% version and documentation can be obtained at:
% http://www.ctan.org/pkg/amsmath





% *** SPECIALIZED LIST PACKAGES ***
%
%\usepackage{algorithmic}
% algorithmic.sty was written by Peter Williams and Rogerio Brito.
% This package provides an algorithmic environment fo describing algorithms.
% You can use the algorithmic environment in-text or within a figure
% environment to provide for a floating algorithm. Do NOT use the algorithm
% floating environment provided by algorithm.sty (by the same authors) or
% algorithm2e.sty (by Christophe Fiorio) as the IEEE does not use dedicated
% algorithm float types and packages that provide these will not provide
% correct IEEE style captions. The latest version and documentation of
% algorithmic.sty can be obtained at:
% http://www.ctan.org/pkg/algorithms
% Also of interest may be the (relatively newer and more customizable)
% algorithmicx.sty package by Szasz Janos:
% http://www.ctan.org/pkg/algorithmicx




% *** ALIGNMENT PACKAGES ***
%
%\usepackage{array}
% Frank Mittelbach's and David Carlisle's array.sty patches and improves
% the standard LaTeX2e array and tabular environments to provide better
% appearance and additional user controls. As the default LaTeX2e table
% generation code is lacking to the point of almost being broken with
% respect to the quality of the end results, all users are strongly
% advised to use an enhanced (at the very least that provided by array.sty)
% set of table tools. array.sty is already installed on most systems. The
% latest version and documentation can be obtained at:
% http://www.ctan.org/pkg/array


% IEEEtran contains the IEEEeqnarray family of commands that can be used to
% generate multiline equations as well as matrices, tables, etc., of high
% quality.




% *** SUBFIGURE PACKAGES ***
%\ifCLASSOPTIONcompsoc
%  \usepackage[caption=false,font=footnotesize,labelfont=sf,textfont=sf]{subfig}
%\else
%  \usepackage[caption=false,font=footnotesize]{subfig}
%\fi
% subfig.sty, written by Steven Douglas Cochran, is the modern replacement
% for subfigure.sty, the latter of which is no longer maintained and is
% incompatible with some LaTeX packages including fixltx2e. However,
% subfig.sty requires and automatically loads Axel Sommerfeldt's caption.sty
% which will override IEEEtran.cls' handling of captions and this will result
% in non-IEEE style figure/table captions. To prevent this problem, be sure
% and invoke subfig.sty's "caption=false" package option (available since
% subfig.sty version 1.3, 2005/06/28) as this is will preserve IEEEtran.cls
% handling of captions.
% Note that the Computer Society format requires a sans serif font rather
% than the serif font used in traditional IEEE formatting and thus the need
% to invoke different subfig.sty package options depending on whether
% compsoc mode has been enabled.
%
% The latest version and documentation of subfig.sty can be obtained at:
% http://www.ctan.org/pkg/subfig




% *** FLOAT PACKAGES ***
%
%\usepackage{fixltx2e}
% fixltx2e, the successor to the earlier fix2col.sty, was written by
% Frank Mittelbach and David Carlisle. This package corrects a few problems
% in the LaTeX2e kernel, the most notable of which is that in current
% LaTeX2e releases, the ordering of single and double column floats is not
% guaranteed to be preserved. Thus, an unpatched LaTeX2e can allow a
% single column figure to be placed prior to an earlier double column
% figure.
% Be aware that LaTeX2e kernels dated 2015 and later have fixltx2e.sty's
% corrections already built into the system in which case a warning will
% be issued if an attempt is made to load fixltx2e.sty as it is no longer
% needed.
% The latest version and documentation can be found at:
% http://www.ctan.org/pkg/fixltx2e


%\usepackage{stfloats}
% stfloats.sty was written by Sigitas Tolusis. This package gives LaTeX2e
% the ability to do double column floats at the bottom of the page as well
% as the top. (e.g., "\begin{figure*}[!b]" is not normally possible in
% LaTeX2e). It also provides a command:
%\fnbelowfloat
% to enable the placement of footnotes below bottom floats (the standard
% LaTeX2e kernel puts them above bottom floats). This is an invasive package
% which rewrites many portions of the LaTeX2e float routines. It may not work
% with other packages that modify the LaTeX2e float routines. The latest
% version and documentation can be obtained at:
% http://www.ctan.org/pkg/stfloats
% Do not use the stfloats baselinefloat ability as the IEEE does not allow
% \baselineskip to stretch. Authors submitting work to the IEEE should note
% that the IEEE rarely uses double column equations and that authors should try
% to avoid such use. Do not be tempted to use the cuted.sty or midfloat.sty
% packages (also by Sigitas Tolusis) as the IEEE does not format its papers in
% such ways.
% Do not attempt to use stfloats with fixltx2e as they are incompatible.
% Instead, use Morten Hogholm'a dblfloatfix which combines the features
% of both fixltx2e and stfloats:
%
% \usepackage{dblfloatfix}
% The latest version can be found at:
% http://www.ctan.org/pkg/dblfloatfix




%\ifCLASSOPTIONcaptionsoff
%  \usepackage[nomarkers]{endfloat}
% \let\MYoriglatexcaption\caption
% \renewcommand{\caption}[2][\relax]{\MYoriglatexcaption[#2]{#2}}
%\fi
% endfloat.sty was written by James Darrell McCauley, Jeff Goldberg and 
% Axel Sommerfeldt. This package may be useful when used in conjunction with 
% IEEEtran.cls'  captionsoff option. Some IEEE journals/societies require that
% submissions have lists of figures/tables at the end of the paper and that
% figures/tables without any captions are placed on a page by themselves at
% the end of the document. If needed, the draftcls IEEEtran class option or
% \CLASSINPUTbaselinestretch interface can be used to increase the line
% spacing as well. Be sure and use the nomarkers option of endfloat to
% prevent endfloat from "marking" where the figures would have been placed
% in the text. The two hack lines of code above are a slight modification of
% that suggested by in the endfloat docs (section 8.4.1) to ensure that
% the full captions always appear in the list of figures/tables - even if
% the user used the short optional argument of \caption[]{}.
% IEEE papers do not typically make use of \caption[]'s optional argument,
% so this should not be an issue. A similar trick can be used to disable
% captions of packages such as subfig.sty that lack options to turn off
% the subcaptions:
% For subfig.sty:
% \let\MYorigsubfloat\subfloat
% \renewcommand{\subfloat}[2][\relax]{\MYorigsubfloat[]{#2}}
% However, the above trick will not work if both optional arguments of
% the \subfloat command are used. Furthermore, there needs to be a
% description of each subfigure *somewhere* and endfloat does not add
% subfigure captions to its list of figures. Thus, the best approach is to
% avoid the use of subfigure captions (many IEEE journals avoid them anyway)
% and instead reference/explain all the subfigures within the main caption.
% The latest version of endfloat.sty and its documentation can obtained at:
% http://www.ctan.org/pkg/endfloat
%
% The IEEEtran \ifCLASSOPTIONcaptionsoff conditional can also be used
% later in the document, say, to conditionally put the References on a 
% page by themselves.




% *** PDF, URL AND HYPERLINK PACKAGES ***
%
\usepackage{url}
% url.sty was written by Donald Arseneau. It provides better support for
% handling and breaking URLs. url.sty is already installed on most LaTeX
% systems. The latest version and documentation can be obtained at:
% http://www.ctan.org/pkg/url
% Basically, \url{my_url_here}.





% *** Do not adjust lengths that control margins, column widths, etc. ***
% *** Do not use packages that alter fonts (such as pslatex).         ***
% There should be no need to do such things with IEEEtran.cls V1.6 and later.
% (Unless specifically asked to do so by the journal or conference you plan
% to submit to, of course. )


% correct bad hyphenation here
\hyphenation{op-tical net-works semi-conduc-tor}


\begin{document}
%
% paper title
% Titles are generally capitalized except for words such as a, an, and, as,
% at, but, by, for, in, nor, of, on, or, the, to and up, which are usually
% not capitalized unless they are the first or last word of the title.
% Linebreaks \\ can be used within to get better formatting as desired.
% Do not put math or special symbols in the title.
\title{CSC8102 System Security Coursework Report}
%
%
% author names and IEEE memberships
% note positions of commas and nonbreaking spaces ( ~ ) LaTeX will not break
% a structure at a ~ so this keeps an author's name from being broken across
% two lines.
% use \thanks{} to gain access to the first footnote area
% a separate \thanks must be used for each paragraph as LaTeX2e's \thanks
% was not built to handle multiple paragraphs
%
%
%\IEEEcompsocitemizethanks is a special \thanks that produces the bulleted
% lists the Computer Society journals use for "first footnote" author
% affiliations. Use \IEEEcompsocthanksitem which works much like \item
% for each affiliation group. When not in compsoc mode,
% \IEEEcompsocitemizethanks becomes like \thanks and
% \IEEEcompsocthanksitem becomes a line break with idention. This
% facilitates dual compilation, although admittedly the differences in the
% desired content of \author between the different types of papers makes a
% one-size-fits-all approach a daunting prospect. For instance, compsoc 
% journal papers have the author affiliations above the "Manuscript
% received ..."  text while in non-compsoc journals this is reversed. Sigh.

\author{Matthew~D~Ball, \{\url{m.d.ball2@ncl.ac.uk}\}}% <-this % stops a space

% note the % following the last \IEEEmembership and also \thanks - 
% these prevent an unwanted space from occurring between the last author name
% and the end of the author line. i.e., if you had this:
% 
% \author{....lastname \thanks{...} \thanks{...} }
%                     ^------------^------------^----Do not want these spaces!
%
% a space would be appended to the last name and could cause every name on that
% line to be shifted left slightly. This is one of those "LaTeX things". For
% instance, "\textbf{A} \textbf{B}" will typeset as "A B" not "AB". To get
% "AB" then you have to do: "\textbf{A}\textbf{B}"
% \thanks is no different in this regard, so shield the last } of each \thanks
% that ends a line with a % and do not let a space in before the next \thanks.
% Spaces after \IEEEmembership other than the last one are OK (and needed) as
% you are supposed to have spaces between the names. For what it is worth,
% this is a minor point as most people would not even notice if the said evil
% space somehow managed to creep in.

% make the title area
\maketitle


% To allow for easy dual compilation without having to reenter the
% abstract/keywords data, the \IEEEtitleabstractindextext text will
% not be used in maketitle, but will appear (i.e., to be "transported")
% here as \IEEEdisplaynontitleabstractindextext when the compsoc 
% or transmag modes are not selected <OR> if conference mode is selected 
% - because all conference papers position the abstract like regular
% papers do.
\IEEEdisplaynontitleabstractindextext
% \IEEEdisplaynontitleabstractindextext has no effect when using
% compsoc or transmag under a non-conference mode.



% For peer review papers, you can put extra information on the cover
% page as needed:
% \ifCLASSOPTIONpeerreview
% \begin{center} \bfseries EDICS Category: 3-BBND \end{center}
% \fi
%
% For peerreview papers, this IEEEtran command inserts a page break and
% creates the second title. It will be ignored for other modes.
\IEEEpeerreviewmaketitle



\IEEEraisesectionheading{\section{File Encryption and Decryption}\label{sec:encryptiondecryption}}
% Computer Society journal (but not conference!) papers do something unusual
% with the very first section heading (almost always called "Introduction").
% They place it ABOVE the main text! IEEEtran.cls does not automatically do
% this for you, but you can achieve this effect with the provided
% \IEEEraisesectionheading{} command. Note the need to keep any \label that
% is to refer to the section immediately after \section in the above as
% \IEEEraisesectionheading puts \section within a raised box.


% The very first letter is a 2 line initial drop letter followed
% by the rest of the first word in caps (small caps for compsoc).
% 
% form to use if the first word consists of a single letter:
% \IEEEPARstart{A}{demo} file is ....
% 
% form to use if you need the single drop letter followed by
% normal text (unknown if ever used by the IEEE):
% \IEEEPARstart{A}{}demo file is ....
% 
% Some journals put the first two words in caps:
% \IEEEPARstart{T}{his demo} file is ....
% 
% Here we have the typical use of a "T" for an initial drop letter
% and "HIS" in caps to complete the first word.
\IEEEPARstart{T}{he} program for part one of this coursework allows a user to
encrypt and decrypt their own files according to a password. These files are
encrypted via AES-128 running in Cipher Block Chaining (CBC) mode and are
signed with a HMAC function. Any encrypted file is comprised of the initial
vector for the CBC operation, the resultant ciphertext, and a signature at the
end which ensures integrity of the data and allows us to confirm that the
password used was the correct one.

The program simply defines an encrypted file as any file that ends with the
\texttt{.8102} extension A file which does not end with that extension is
assumed to not be encrypted at all. In order to build this program, the user
needs to run the two maven commands: \texttt{mvn clean install} and
\texttt{mvn jar:jar} to create the resultant \texttt{.jar} file inside the
\texttt{target/} folder. From there, the user can use commands to encrypt and
decrypt those files.

A better explanation of the commands and their use can be found in the
\texttt{README.md} file packaged with the code. If the file you are trying to
encrypt or decrypt does not exist, then the program will complain and exit
once it detects that. The same happens if the program detects that you
are trying to encrypt a file with the \texttt{.8102} extension, or if you are
trying to decrypt a file without it. Once the input file has been encrypted or
decrypted, it will be deleted and a new file with the appropriate file
extension will be written to.

The encrypted file will contain Base64 encoded text instead of direct binary
in order to act as an ASCII Armor file and avoid writing directly to binary
due to the problems that it can cause later down the line.

\subsection{Implementation}

Let us say that we wish to encrypt the file within Figure
\ref{fig:secretinformation} that contains some confidential data using this
program.

\begin{figure}[h!]
	\centering
	\includegraphics[width=0.8\linewidth]{images/secret_information}
	\caption{A dangerous secret}
	\label{fig:secretinformation}
\end{figure}

We would use the \texttt{-e [filename]} flag to encrypt this file and be
presented with the screen in Figure \ref{fig:secretpassword}, asking us to
enter a password for the file.

\begin{figure}[h!]
	\centering
	\includegraphics[width=0.9\linewidth]{images/secret_password}
	\caption{The password for a dangerous secret}
	\label{fig:secretpassword}
\end{figure}

Once the password has been entered, the program goes ahead and encrypts the
file by first hashing the given password with SHA-256 and splitting it into two
16-byte halves. It designates the first half as the AES-128 key we are going to
use later and the second half as a MAC code we use as input for the HMAC
function later on.

To encrypt the plaintext, we first securely generate 16 bytes to act as an
initial vector (IV) for our CBC engine. Then we give the IV, the AES-128 key
generated in the last part, and the plaintext to the encryption libraries with
instructions to encrypt using AES with CBC and using the PKCS5 padding scheme.
This generates the ciphertext equivalent of the plaintext that can be decrypted
when provided along with the IV and password used to create it.

In order to provide integrity to the data (and to provide a way to validate
that the user provided the correct password during decryption), we then combine
the IV and the ciphertext, then hash them down to a 20-byte signature using a
secure HMAC function with SHA1 (Even though SHA1 has been proven insecure
\cite{Wang2005}), using the MAC code generated above as an input. By doing this,
we ensure that if the user enters a different password, then the MAC code
generated from that password will result in a different signature for the file.

At this point, we concatenate the IV, the ciphertext, and the signature into a
new \texttt{.8102} file that we replace the original file with. The new file is
encoded with Base64 and looks like the file in Figure
\ref{fig:secretencryption}.

\begin{figure}[h!]
	\centering
	\includegraphics[width=0.9\linewidth]{images/secret_encryption}
	\caption{The encryption of a dangerous secret}
	\label{fig:secretencryption}
\end{figure}

Decryption of the file is done with the \texttt{-d [filename].8102} flag which
then present the screen in Figure \ref{fig:secretdecryption}, asking for the
password again.

\begin{figure}[h!]
	\centering
	\includegraphics[width=0.9\linewidth]{images/secret_decryption}
	\caption{The decryption of a dangerous secret}
	\label{fig:secretdecryption}
\end{figure}

At which point, the program reads in that file and splits apart the input into
the first 16 bytes which are the IV, the 20 bytes at the end which are the
signature for the file, and the rest in the middle which is the ciphertext.
Since we have everything that was initially used to create the signature during
the encryption stage, we now create our own version of the signature with the
MAC code from the password hash, the IV, and the ciphertext and compare it with
the signature attached to the file.

\begin{figure}[h!]
	\centering
	\includegraphics[width=0.9\linewidth]{images/secret_failure}
	\caption{A failure at decrypting our dangerous secret}
	\label{fig:secretfailure}
\end{figure}

If the hash that was created from the password and the contents of the file
does not match the one attached to the file then we stop right there and report
back to the user that the password was either incorrect or the file integrity
has been broken (see Figure \ref{fig:secretfailure}). If the signatures matched,
we decrypt the ciphertext and write the plaintext to the original file,
deleting the encrypted file afterwards.

\subsection{Future Improvements}

Due to the nature of Java, this program can be exported as a library to other
applications that may wish to include encryption and decryption in their
programs. However, a few improvements would probably need to be made for this.
One such improvement could be that the file extension becomes custom for the
program and that, if a user changes the \texttt{.8102} extension to something
else, then the program will still attempt a decryption if the file is long
enough.

Another potential improvement for this program in terms of usability would be
to create a graphical interface that is loaded instead of the command line.
This would help to remove the errors that happen when the files that the user
provides are not found or are not fit for encryption or decryption. It would
also allow for some advanced user feedback where the program reports back to
the user on how successful or unsuccessful the operations were.

\section{Hash Cracking}\label{sec:hashcracking}

The program for part two of this coursework is a hash cracking application
which allows a user to brute force eight different passwords that were hashed
using SHA-256. It does this by performing an exhaustive search against a number
of different dictionaries. We say that there are five different dictionaries
available for this exhaustive search:

\begin{enumerate}
	\item One of the ten most popular girl names in England (lower-case).
	\item One of the ten most popular boy names in England (lower-case).
	\item A name specified in the first two dictionaries but with no specified
	case and a number between 1 and 9999 inclusive afterwards.
	\item A lower-case English word found in Moby Dick.
	\item A combination of four alphanumeric characters of no specified case
	and a few special characters.
\end{enumerate}

The program takes two arguments: an input file that contains hashes the user
wants to crack, and an output file where whatever results from the cracking
will be placed. In order to build this program, the user needs to run the two
maven commands: \texttt{mvn clean install} and \texttt{mvn jar:jar} to create
the resultant \texttt{.jar} file inside the \texttt{target/} folder. From that,
the user can use the program to attempt to crack several hashes.

A more complete explanation of the command can be found in the
\texttt{README.md} which is packaged with this program. If the user does not
provide both the \texttt{-i [input hashes]} and \texttt{-o [output file]}
flags, then the program will not run. The same thing happens if the user
provides a non-existent input hash file. After execution, the program will
create and fill the output file with the results from the hash cracking. If the
output file already exists, then it will be overwritten with the new outputs.

The output file will contain each cracked hash on a new line with the password
used to crack it in the format of: \texttt{Hash: [hash] Password: [password]}.
Typical execution of the program takes less than thirty seconds.

\subsection{Implementation}

The coursework provides the hashes seen in Figure \ref{fig:crackedhashes} to be
cracked. Each hash must be on a new line and encrypted in SHA-256. They must
also be converted from binary into hexadecimal strings so that they can be read
in.

\begin{figure}[h!]
	\centering
	\includegraphics[width=0.8\linewidth]{images/cracked_hashes}
	\caption{A set of hashes to be cracked}
	\label{fig:crackedhashes}
\end{figure}

Running the command \texttt{-i [input] -o [output]} with that file as an input
will run the program and report back once we have either exhausted all
dictionaries which are defined in Section \ref{sec:hashcracking} or we have
solved all of the hashes. In this case (as seen in Figure
\ref{fig:crackedcommands}), we return once we have solved all eight hashes and
provide some metrics as to the running of the program.

\begin{figure}[h!]
	\centering
	\includegraphics[width=0.9\linewidth]{images/cracked_commands}
	\caption{A command which was run to crack hashes}
	\label{fig:crackedcommands}
\end{figure}

This command hides a large amount of complexity in generation and testing. As
seen in Figure \ref{fig:crackedcommands}, over 35 million password comparisons
were performed using the dictionaries that we have defined. The program
generates most of those passwords at runtime. A test was performed during
development to see whether performance improvements were observed when a large
list of pre-generated passwords was provided to the program instead. This test
resulted in a runtime that was twice that of the runtime observed above,
possibly due to the inefficiencies of opening files in Java.

The generation of password combinations in the program has been optimised to
reduce the amount of redundant processing and keep the memory consumption to a
minimum. This is done by generating the base strings for password combinations
during initial generation, then adding on the specific iteration later down the
line when the password is needed.

\begin{figure}[h!]
	\centering
	\includegraphics[width=0.9\linewidth]{images/cracked_passwords}
	\caption{A set of hashes that were cracked}
	\label{fig:crackedpasswords}
\end{figure}

All of which eventually results in the file in Figure
\ref{fig:crackedpasswords} which shows the output from the program. The
passwords that corresponded to the hashes in \ref{fig:crackedhashes} are as
follows:

\begin{itemize}
	\item Hash: \texttt{5e0176c...} Password: \texttt{sophie}
	\item Hash: \texttt{b9dd960...} Password: \texttt{charlie}
	\item Hash: \texttt{ab8a711...} Password: \texttt{inopithelioma}
	\item Hash: \texttt{4abb806...} Password: \texttt{unbreakable}
	\item Hash: \texttt{d813086...} Password: \texttt{JeSsicA42}
	\item Hash: \texttt{ffff0e6...} Password: \texttt{haRRY1980}
	\item Hash: \texttt{20abfe9...} Password: \texttt{K4t!}
	\item Hash: \texttt{20d9787...} Password: \texttt{@c3*}
\end{itemize}

\subsection{Future Improvements}

Improvements to this program can only really come in two ways: increasing the
size of the dictionary from which we can draw passwords, and optimising the way
we generate and compare passwords so that this program can scale as we increase
the number of options in the dictionary. Part of the problem in doing the former
is that at some point, we run into problems with memory and time. It takes time
and space to generate and store a string of four alphanumeric characters in RAM
during program execution and most applications these days place a minimum
character limit of around six characters.

If we say that all the characters we used are valid, it is the difference
between $71^{4}$ combinations and $71^{6}$ combinations which is roughly 128
billion more combinations that need to be tested. This would be a true brute
force attack against those passwords and would only consider passwords of six
characters long. While attacking passwords of low length and predictable
contents is relatively easy, once the passwords reach a certain length and
entropy then attacking them becomes unfeasible.

% references section

% can use a bibliography generated by BibTeX as a .bbl file
% BibTeX documentation can be easily obtained at:
% http://mirror.ctan.org/biblio/bibtex/contrib/doc/
% The IEEEtran BibTeX style support page is at:
% http://www.michaelshell.org/tex/ieeetran/bibtex/
\bibliographystyle{IEEEtran}
\bibliography{IEEEabrv,references}
% argument is your BibTeX string definitions and bibliography database(s)
%\bibliography{IEEEabrv,../bib/paper}
%
% <OR> manually copy in the resultant .bbl file
% set second argument of \begin to the number of references
% (used to reserve space for the reference number labels box)

% that's all folks
\end{document}


